%\documentclass[conference]{IEEEtran}
%\IEEEoverridecommandlockouts
%\documentclass[sigconf]{acmart}
%\let\Bbbk\relax %% fix bug
\documentclass[twocolumn,10pt]{article}
\usepackage[utf8]{inputenc}

% =======================
\usepackage{amssymb}
\usepackage{amsmath}
\usepackage{amstext}
\usepackage{amsopn}
%\usepackage{algorithmic}
\usepackage{graphicx}
\usepackage{textcomp}
\usepackage{xcolor}

\usepackage{textcomp}

\usepackage{boxedminipage}
\usepackage{enumerate}
\usepackage{multirow}
\usepackage{url}
\usepackage{times}
\usepackage{version}
% \usepackage[pdftex]{graphicx}
\usepackage{epsfig}
\usepackage{epsf}
%\usepackage{graphics}
\usepackage{caption}
\usepackage{subfigure}
\usepackage{algorithm}
\usepackage{algpseudocode}
%\PassOptionsToPackage{bookmarks={false}}{hyperref}
%%%%%%%%%%%%
\usepackage{comment}
\usepackage{multicol}
\usepackage{booktabs}
\usepackage{dblfloatfix}
% ==========================
%\usepackage[a4paper, margin = 2.5cm]{geometry}

\AtBeginDocument{%
  \providecommand\BibTeX{{%
    \normalfont B\kern-0.5em{\scshape i\kern-0.25em b}\kern-0.8em\TeX}}}

\begin{document}

\title{The Accuracy of KNN, decision tree, random forest, SVM, neural network, naive Bayes classifier and PLA for early Prediction of Diabetes}

\author{Hsieh Cheng-Han}
\date{October 2022}
\maketitle

\section*{abstract}
  This work compares the accuracy of some classifiers for early the prediction of diabetes. More specifically, 
  the research compares the accuracy of k-nearest neighbors (KNN) algorithm, decision tree, random forest, 
  support vector machine (SVM), neural network, naive Bayes classifier, and perceptron learning algorithm (PLA) on the prediction of diabetes, which 
  the dataset is collected with eight features, times of pregnancy, concentration of glucose in blood, blood 
  pressure, skin thickness,  concentration of insulin in blood, body mass index (BMI), the value of diabetes 
  pedigree function and age.

  The result show that XXX is the most accurate on the prediction of diabetes.

\section{Introduction}
\label{sec:Introduction}
  Diabetes is a chronic disease which may cause many complications. There're lots of reason that can put a person 
  at the highly risk of having diabetes, such as age, obesity, lack of exercies, and more on. So many reasons 
  interweave together making the manual prediction on diabetes is nearly impossible. However, lots of works \cite{MUJUMDAR2019292} \cite{MAHBOOBALAM2019100204} \cite{10.3389/fgene.2018.00515}
  show that it is possible to have high accuracy by using machine learning techniques, such as random forest, 
  K-means clustering, neural network, and so on. 

  By collecting the essential data of human body, prediction of diabetes can be turn into classification problem. 
  Imagine that an individual case with essential data is a point in hyperspace, if it is closer to the cluster
  having diabetes, this case is more likely to have diabetes in the future, otherwise, this case is more likely 
  healthy. But there are lots of machine learning techniques born to solve classification problem, it remains a 
  problem that which technique having the hightest accuracy on the prediction of diabetes. 

  To find out which techniques is more suitable to predict diabetes, this work examines the diagnosis of diabetes 
  using KNN algorithm, decision tree, random forest, SVM, neural network naive Bayes classifier, and PLA.
\section{Related works}
\label{sec:Related works}
  \bf{k-nearest neighbors (KNN) classification algorithm}: \rm{The} KNN classification algorithm is a supervised learning 
  method which is first developed by Fix and Hodges \cite{10.2307/1403797}. The idea of KNN is based on the idiom, 
  "birds of a feather flock together". By picking the $k$-nearest neighbors of a data point, the unkonwn class label 
  can be determined. Lots of works \cite{6528591} \cite{8276012} \cite{vijayan2014study} show the fact that KNN performs  
  well for prediction of diabetes disease.

  \bf{decision tree}: \rm{Unlike} KNN uses distance to determine the outcome, 
  decision tree uses a sequence of deicsion that maximize the information gain, which can distinguish the class label of 
  data as much as possible, to determine the outcome. Many works \cite{5893838} \cite{8342938} have applied deicsion tree 
  method and gain a good accuracy. The advantage of decision tree is fast, easy to implement, and the decision is clear. 
  But the disadvantage is that it is very likely over-fitting and the structure of tree will become more complex with the 
  more the class labels. To solve this problem, the following techniques is developed:

  \bf{random forst}: \rm{Instead} of a signle decision tree, random forest use lots of decision trees, which form a "forest". 
  The decision trees are constructed by random subset of dataset. The key differs random forest from decision tree is that 
  while decision trees consider all the possible feature splits, random forests only select a subset of those features, which 
  reduce the risk of overfitting, bias, and overall variance. In \cite{7972337} \cite{10.1007/978-981-16-2164-2_19}, random 
  forest shows that it can greatly reduce the problem of over-fitting of the single decision tree, and gain an ever higher accuracy.

  \bf{support vector machine (SVM)}: 
  \bf{neural network}: 
  \bf{naive Bayes classifier}: 
  \bf{perceptron learning algorithm (PLA)}: 

\section{KNN}
  The rough process of KNN is described as follow: suppose that there is a dataset which contain $N$ data point, denoted as 
  $(X_i,Y_i)$ where $X_i$ is the features of the i-th individuals data and $Y_i$ is the class label of it. Now 
  a data with unknown class label is given, denoted $(X, Y)$. By a preset distance function $d(P, Q)$, ordering the dataset 
  as $(X_{(1)}, Y_{(1)}), (X_{(2)}, Y_{(2)}), \cdots, (X_{(N)}, Y_{(N)})$ where $d(X_{(1)}, X)\leq d(X_{(2)}, X)\leq\cdots\leq d(X_{(N)}, X)$. 
  Pick the $k$-first class labels to determine the unknow class label, $Y$.
  

\section{decision tree}
  Given a dataset, $D$, which contains $N$ data and class label, the construction of a decision tree can be described as 
  below: suppose there are $M$ candidate decisions, denoted $f_i$. A decision can separate dataset $D$ into $m$ kinds, denote 
  $D'_j$. The decision tree will adopt $\max_{f} G(D, f)=I(D)-\sum^m_{j=1}\frac{N'_j}{N}I(D'_j)$ as node decision, 
  and then recursively construct the tree until the data in separated dataset have the same class label. 
  When a data with unknown class label comes, a decision tree determines recursively by the decision node until the leaf node.
  The function of calculating information, $I$, can be various from implementation. The most famous two information function is 
  entropy, and gini impurity. The formula of information entropy is $I_H(X)=-\sum_{x}p(x)\log_2p(x)$ 
  and gini impurity is $I_G(X)=1-\sum_{x}p(x)^2$.

\section{random forest}
  Given the fact that a signle decision tree can be easily over-fitting, a technique called "random forest" is developed. By 
  constructing $m$ decision trees with randomized subsets of training dataset, the different decision trees forms a "forest". 
  The process that random forest determine a data with unknown class label can be outlined as follow: suppose for a coming data, 
  $c_i$ decision trees in a random forest classify it as class $i$. Then the random forest will classify the data as class $\arg\max_{i} c_i$.


\section{SVM}

\section{neural network}

\section{Naive Bayes classifier}

\section{PLA}

\section{Experiment result}

\section{Conclusion}

\bibliographystyle{IEEEtran}
\bibliography{main}
\end{document}

